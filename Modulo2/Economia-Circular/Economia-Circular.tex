% This is samplepaper.tex, a sample chapter demonstrating the
% LLNCS macro package for Springer Computer Science proceedings;
% Version 2.21 of 2022/01/12
%
\documentclass[runningheads]{llncs}
%
\usepackage[T1]{fontenc}
% T1 fonts will be used to generate the final print and online PDFs,
% so please use T1 fonts in your manuscript whenever possible.
% Other font encondings may result in incorrect characters.


% tightlist command for lists without linebreak
\providecommand{\tightlist}{%
  \setlength{\itemsep}{0pt}\setlength{\parskip}{0pt}}




\usepackage{graphicx}
% Used for displaying a sample figure. If possible, figure files should
% be included in EPS format.
%
% If you use the hyperref package, please uncomment the following two lines
% to display URLs in blue roman font according to Springer's eBook style:
\usepackage{hyperref}
\usepackage{color}
\renewcommand\UrlFont{\color{blue}\rmfamily}


\begin{document}


\title{Flujos de información, adaptación y emergencia en sistemas
circulares de
suministros\thanks{Lydia Bals, Wendy L. Tate, Lisa M. Ellram}}
%
\titlerunning{Cadenas de Suministros en Economía Circular}
% If the paper title is too long for the running head, you can set
% an abbreviated paper title here
%
\author{Ian Johnson\inst{1}\orcidID{0009-0001-2786-9385} \and Carolina
Fernández\inst{1}\orcidID{0009--0004-4219-1041} \and Julián
Hidalgo\inst{1}\orcidID{0009-0002-0617-6800}}


\authorrunning{Grupo Los Ignorantes}
% First names are abbreviated in the running head.
% If there are more than two authors, 'et al.' is used.
%

\institute{Grupo Los Ignorantes}

\maketitle              % typeset the header of the contribution
%
\begin{abstract}
Este documento resume el capítulo 3 del libro ``Circular Economy Supply
Chains''. El capítulo estudia el concepto de ``sistema circular de
suministros'' y la forma en que los flujos de información entre sus
agentes facilita su crecimiento. Finalmente se analizan los resultados
del flujo de información, que se resumen en nuevos agentes y nuevos
patrones espacio-temporales.

\keywords{Economía circular \and Cadena de suministros circular \and Red
de suministros \and Sistemas adaptativos complejos \and Flujos de
información \and Emergencia}

\end{abstract}

\hypertarget{introducciuxf3n}{%
\section{Introducción}\label{introducciuxf3n}}

Se introduce la economía circular mediante una analogía con la cadena de
suministros de la industria papelera en 1820, en la cual la escasez de
fibras hacía que las empresas dependieran de los restos de prendas que
sus compradores pudieran aportar. En este sentido, la economía circular
se define como
\textbf{la forma de recircular materiales mediante técnicas como reutilización, rediseño y reciclaje.}\\

Sin embargo, la razón para impulsar la economía circular en el contexto
moderno ya no es la escasez, sino la forma en que el
\textbf{exceso de consumo} genera pérdidas económicas y daños al
medioambiente a través del aumento de flujos de desechos o la
sobreexplotación de recursos naturales.\\

Se entiende que el motivo para avanzar hacia una economía circular es la
respuesta a estas señales, y se explica que las empresas modernas no
tienen control total sobre la cadena de suministros, sino que dependen
de que otros actores quieran involucrarse en la economía circular. Este
capítulo intenta explicar el modo en que las señales de información
pueden fomentar el desarrollo de la economía circular.

\hypertarget{apariciuxf3n-de-nuevos-modelos-de-sistema}{%
\section{Aparición de nuevos modelos de
sistema}\label{apariciuxf3n-de-nuevos-modelos-de-sistema}}

Se define al \textbf{sistema de suministros} como el sistema que incluye
todos los actores relevantes en la producción (principalmente actores
externos), mientras que la cadena directa de productores de bienes se
define como \textbf{red de suministros}

\hypertarget{cadenas-de-suministros-como-sac.}{%
\subsection{Cadenas de suministros como
SAC.}\label{cadenas-de-suministros-como-sac.}}

Un SAC (sistema adaptativo complejo) es
\textbf{un sistema que contiene una gran cantidad de agentes que interactúan, se adaptan y aprenden}.
En este tipo de sistemas los agentes no se relacionan linealmente, sino
que siguen determinados patrones de comportamiento. Estos son tácitos y
surgen debido a la interacción independiente entre agentes (sin
intervención de una autoridad central), en lo que se conoce como el
\textbf{mecanismo de auto-organización.}\\

Un sistema de suministros se considera como un SAC debido a las
relaciones no lineales entre sus agentes. Los comportamientos clásicos
de un SAC incluyen la capacidad de respuesta de los agentes, la
interconexión y la evolución conjunta con el entorno. En una red de
suministros, los agentes no se conectan con toda la red ni tienen
influencia absoluta sobre ella, sino que toman información de sus
compradores y suministradores locales para tomar decisiones. La red de
suministros también es afectada por factores externos.

\hypertarget{informaciuxf3n-necesaria-para-la-adaptaciuxf3n-de-cadenas-de-suministro.}{%
\subsection{Información necesaria para la adaptación de cadenas de
suministro.}\label{informaciuxf3n-necesaria-para-la-adaptaciuxf3n-de-cadenas-de-suministro.}}

La adaptación surge a partir de acciones condicionales en respuesta a
flujos de información. En un SAC, un agente A toma una decisión y otro
agente B responde mediante una decisión condicional en base a la
información recibida. Al acumular las decisiones de todos los agentes se
construye una estructura dinámica que cambia junto con los
comportamientos de los agentes de la red y el entorno. Los flujos de
información deben ser transmitidos en tiempo y forma para evitar
acciones erróneas o subóptimas por parte del agente B.\\

El flujo de información crea una dependencia entre ambos actores, ya que
B busca su punto óptimo de funcionamiento para la información que le
brindan A y el entorno, a la vez que busca comprender el patrón de
comportamiento que siguen sus de decisión.\\

Algunas definiciones le dan al flujo de información una trascendencia
similar a los flujos de dinero o bienes materiales, ya que, incluso si
los flujos de información fueran incompletos o deficientes, las empresas
se ven obligadas a basar sus decisiones en ellos. Un flujo ideal de
información ayuda al éxito de la cadena de suministros, ya que contiene
datos actualizados y precisos. Por lo tanto, se dice que las empresas
mejoran su esquema de decisión en base a la calidad de la información
recibida.

\hypertarget{desde-la-adaptaciuxf3n-a-la-emergencia}{%
\subsection{Desde la adaptación a la
emergencia}\label{desde-la-adaptaciuxf3n-a-la-emergencia}}

Como ningún agente tiene control sobre todos los demás agentes de una
red, los patrones de comportamiento de la red surgen a partir del modo
en que agentes individuales se adaptan a sus flujos de información
locales. Así, toda la estructura de la cadena de producción se construye
a partir de la capacidad de respuesta de cada agente del sistema.\\

Además de tener capacidad de respuesta, los agentes de una red de
suministros deben controlar aquellos aspectos que estén a su alcance
para evitar desbalances en el resto de la cadena productiva.

\hypertarget{flujos-de-informaciuxf3n-para-la-producciuxf3n-circular}{%
\section{Flujos de información para la producción
circular}\label{flujos-de-informaciuxf3n-para-la-producciuxf3n-circular}}

Estas dinámicas CAS pueden ayudarnos a entender los tipos de información
necesaria para que surjan sistemas circulares de suministro.
Específicamente, los actores necesitan información de (1) redes locales
de suministro, (2) sistemas extendidos de suministro, e idealmente
también (3) la biósfera, de acuerdo con la definición de economía
circular de Murray ``maximizar el funcionamiento del ecosistema''.
Necesitan continuamente tales flujos de información y capacidad
adaptativa para reaccionar, ya que no hay equilibrio único en ningún
CAS. Los sistemas están continuamente evolucionando, significando que
los actores necesitan aprender y adaptarse a las condiciones cambiantes
a medida que vienen (Gunderson \& Holling, 2002). Por lo tanto,
necesitamos información continuamente para adaptarnos.

\hypertarget{redes-de-suministros-locales}{%
\subsection{Redes de suministros
locales}\label{redes-de-suministros-locales}}

Las redes locales circulares de suministro reducen el consumo de energía
necesario para transporte y hacen más fácil monitorear los impactos
sistémicos, por lo que son prioritarias en regiones geográficas
limitadas (Korhonen, 2020). Dos tipos clave de información local son
necesarios: que materiales de desecho están disponibles, y que puede
hacerse con ellos. Estos flujos de información toman dos roles descritos
por Tate et al.~(2019) como carroñeros y descomponedores. Los carroñeros
son ``compañías que se alimentan de recursos de desecho de otras
compañías en el sistema, redistribuyendo recursos de regreso al sistema
que las compañías pueden reutilizar''. Los descomponedores, por otro
lado, son ``compañías que usan recursos de desecho de los productores,
consumidores y carroñeros''. Una fuente de materiales de desecho para
carroñeros son los fabricantes locales, que pueden tener desechos de
flujo lateral que otras firmas podrían usar (Wells \& Seitz, 2005). Se
necesitan flujos de información para entender qué materiales están
disponibles, así como las cantidades y ritmos con que los residuos son
producidos. Las cantidades limitadas pueden afectar la capacidad de las
empresas para escalar la producción circular (Patala, Albareda \& Halme,
2018; Sprecher et al., 2017). Sin embargo, las compañías
tradicionalmente no comparten información sobre sus corrientes de
desechos, por miedo a revelar información sensible (Patala et al.,
2018). La proximidad geográfica cercana puede facilitar la colaboración
(Prendeville, Cherim \& Bocken, 2018) e incluso aliviar potencialmente
problemas de confianza (Patala e al., 2018). Los sistemas de simbiosis
industrial, que son típicamente conjuntos de empresas delimitadas
geográficamente que ingresan los desechos de otros como materia prima
para sus propios procesos (Chertow, 2000), pueden facilitar una
confianza más fuerte, integración, e información compartida entre firmas
(Lopes de Sousa Jabbour et al., 2019). La cuestión de la disponibilidad
y previsión de los residuos es incluso más difícil para compañías usando
desechos postconsumo, ya que hay poca información sobre estos en
específico. Sin datos sobre cantidades de desechos disponibles, las
empresas podrían terminar teniendo muy poco o demasiado del material que
necesitan. Los carroñeros en un sistema de suministro circular pueden
también ayudar a superar una de las barreras claves para cambiar hacia
la circularidad: la logística inversa para llevar los materiales de
desecho a un lugar centralizado para el procesamiento (Farooque et al.,
2019). El cambio a los sistemas circulares de suministro está abriendo
nuevos caminos para que otras firmas comiencen a asumir este rol a
través de nuevos modelos comerciales de reciclaje (Lüdeke-Freund, Gold
\& Bocken, 2019; Patala et al., 2018). Un ejemplo son los sistemas de
recolección de residuos geográficamente descentralizados, como los
programas de devolución de botellas en el punto de venta que usan
escáneres y datos de productos (Miller, 2018). En lugares donde no
existe una infraestructura de gestión de residuos bien desarrollada, el
papel de recolector también puede ser desempeñado por individuos que
venden los residuos recolectados a instalaciones centralizadas
(Troschinetz y Mihelcic, 2009). Para utilizar los materiales
recolectados, los descomponedores primero necesitan información sobre su
composición. En el caso de la remanufacturación, también necesitan
información sobre la vida útil del producto para ayudar a evaluar su
estado (Laubscher y Marinelli, 2014). Los productores pueden ayudar a
los recolectores y descomponedores poniendo a su disposición información
sobre el producto, lo que hace que la recolección, clasificación y
producción secundaria sean más factibles (Jabbour, Jabbour, Sarkis y
Filho, 2017; Laubscher y Marinelli, 2014). Nuevas tecnologías como la
identificación por radiofrecuencia (RFID) y el blockchain son
prometedoras para permitir ideas como los pasaportes de materiales
(Garcia-Torres, Albareda, Rey-Garcia y Seuring, 2019; Tate et al.,
2019). También se necesita información sobre lo que pueden hacer con los
materiales residuales. Aunque las prácticas modernas de reciclaje para
algunos materiales existen desde hace décadas, los científicos de
materiales todavía están estudiando cómo dar una nueva vida a otros
materiales y como pueden desarrollar nuevas alternativas de materiales
que sean más propicias para la circularidad.

\hypertarget{sistemas-de-suministros-extendidos}{%
\subsection{Sistemas de suministros
extendidos}\label{sistemas-de-suministros-extendidos}}

Un flujo crítico es la información de los gobiernos a las redes de
suministro sobre sus iniciativas circulares. Los gobiernos pueden
desempeñar un papel importante incentivando la aparición de redes de
suministro circular a través de políticas amigables con la circularidad,
financiamiento, investigación, desarrollo de asociaciones y/o estándares
de adquisición pública (de Jesus \& Mendonça, 2018; Masi, Day, \&
Godsell, 2017; Patala et al., 2018; Prendeville et al., 2018; Winans et
al., 2017). Podrían ir aún más lejos al facilitar proactivamente la
economía circular a través de estándares de productos, como la facilidad
de desmontaje (Vanegas et al., 2017), o mediante el lanzamiento de
experimentos circulares a nivel de la ciudad (Prendeville et al., 2018).
Las regulaciones sobre residuos y lo que se puede hacer con ellos, y la
retirada del financiamiento previamente ofrecido, pueden ser barreras
para la aparición de redes de suministro circular (de Jesus \& Mendonça,
2018; Winans et al., 2017). Otro flujo relevante es sobre las
proyecciones de escasez de materiales críticos, lo que podría ser un
impulsor hacia la circularidad. Pueden ser impulsadas por la disminución
de las cantidades que se pueden extraer o por interrupciones temporales
en las cadenas de suministro (Gaustad, Krystofik, Bustamante y Badami,
2018). Por último, los minoristas que buscan adoptar la economía
circular también necesitan comprender la disposición de los consumidores
a comprar sus productos circulares o traer su propio desperdicio para
ser reutilizado. Uno de los problemas clave es la percepción de que los
productos hechos de materiales de desperdicio son de menor calidad, lo
que puede impedir que los bienes circulares desplacen a los no
circulares. Prometedoramente, parece que la demanda de productos
circulares ha estado mejorando, lo que podría mitigar el riesgo de este
tipo de ``rebote de la economía circular'' (Zink \& Geyer, 2017). En
algunos sistemas circulares, los consumidores también pueden ser la
fuente de materiales de desecho para los recolectores. Los consumidores
necesitan comprender dónde encontrar a los recolectores, o cómo
convertirse ellos mismos en recolectores así como cualquier requisito de
material: por ejemplo si necesitan ser limpiados o desmontados antes de
llevarlos a un recolector. Además, lograr que los consumidores se sumen
a iniciativas circulares podría potencialmente crear un entorno político
más favorable, al menos en sistemas democráticos.

\hypertarget{impactos-en-la-biuxf3sfera}{%
\subsection{Impactos en la biósfera}\label{impactos-en-la-biuxf3sfera}}

Finalmente, los sistemas de suministro circular deben tener flujos de
información que les permitan comprender y adaptarse a los impactos de la
humanidad en la biósfera. Las empresas necesitan cambiar a una nueva
lógica que centre la sostenibilidad ecológica a largo plazo en la toma
de decisiones (Montabon, Pagell y Wu, 2016). Necesitan flujos de
información sobre indicadores biofísicos a múltiples escalas y la
capacidad de tomar decisiones basadas en ellos. Se incorporan los
indicadores de los límites planetarios en las evaluaciones del ciclo de
vida o en nuevas herramientas ``para facilitar la evaluación
predictiva'' (Clift et al., 2017, p.~4). Los actores necesitan
comprender cómo trabajar dentro de cantidades y tasas de extracción y
descarte de recursos, apropiadas para la biosfera. Esto es importante
para la gestión tanto de recursos biológicos como tecnológicos, ya que
los primeros ``perturbarán seriamente nuestros ecosistemas'' si no se
liberan ``cuidadosamente, a un ritmo en resonancia con el orden
natural'' (Skene, 2018, p.~485). Se debe contar con organizaciones en
los sistemas de suministro circular más amplios que puedan rastrear y
operacionalizar información a diferentes escalas. Los actores
gubernamentales que monitorean estas escalas pueden incorporar datos en
incentivos y políticas, y las organizaciones que monitorean cambios
locales y más rápidos también pueden informar sobre los impactos en el
terreno.

\hypertarget{posible-emergencia-en-suministros-circulares}{%
\section{Posible emergencia en suministros
circulares}\label{posible-emergencia-en-suministros-circulares}}

El texto en un inicio habla sobre cómo pueden surgir nuevas estructuras
y patrones en sistemas circulares de suministro si los actores eligen
adaptarse a ellos. Se mencionan dos tipos de surgimiento en particular:
nuevos roles y redes de actores, y nuevos patrones espaciales y
temporales. Se explica cómo, a través de la recepción de señales de
información específicas, pueden surgir nuevos roles y redes de
producción circular y cómo estos se relacionan con las condiciones
ambientales y gubernamentales. Se utiliza el siguiente ejemplo para
ilustrar esto:\\

Para ilustrar esto, podemos volver al caso de la empresa textil
Finlayson, que produce alfombras de trapo con las sábanas viejas de sus
clientes: Las alfombras de trapo son una artesanía tradicional
finlandesa que ha desaparecido en gran medida en los últimos años.
Finlayson quería revivir esta tradición y al mismo tiempo abordar el
problema de los residuos textiles en Europa. En la primavera de 2016,
realizaron su primera campaña de recolección de sábanas viejas de los
clientes y recibieron 11 toneladas de ellas. A cambio, los clientes
recibieron descuentos en sus próximas compras para demostrar que ``las
sábanas viejas tienen valor''. Debido a que la producción de alfombras
de trapo había disminuido en popularidad, Finlayson no pudo encontrar
una máquina para hacerlas, pero encontró un pequeño proveedor en
Finlandia que ahora trabaja a tiempo completo gracias al contrato de
Finlayson.\\

En este ejemplo, la red de suministro para las alfombras de trapo surgió
a través de la autoorganización. Finlayson difundió información sobre la
recolección de sábanas y los incentivos a los clientes dentro de su
sistema de suministro extendido. Los clientes que se adaptaron a esa
información al llevar sus sábanas se convirtieron en parte de la red de
suministro directa. En el lado de la fabricación, Finlayson encontró, a
través de alguna señal de información, uno de los últimos productores
industriales de alfombras de trapo en Finlandia. Gracias al contrato de
Finlayson, surgieron nuevos patrones temporales (trabajando a tiempo
completo) y estructuras de recursos (empleando más personal) dentro de
ese proveedor. Toda la iniciativa de las alfombras de trapo fue
impulsada por dos piezas de información del sistema de suministro
extendido: (1) el problema de los residuos textiles en Europa y (2) la
información implícita de que habría un mercado para revivir las
tradicionales alfombras de trapo finlandesas.

\hypertarget{nuevos-patrones-espaciales-y-temporales}{%
\subsection{Nuevos patrones espaciales y
temporales}\label{nuevos-patrones-espaciales-y-temporales}}

En esta parte, se habla sobre cómo en los sistemas de suministro
circular pueden surgir nuevos patrones y estructuras si los actores se
adaptan a la información que reciben. Se pueden dar dos tipos de
surgimiento: nuevos roles y redes de actores, y nuevos patrones
espaciales y temporales. En cuanto a los patrones espaciales, pueden
surgir sistemas de suministro locales debido a la capacidad de los
actores de monitorear y confiar entre sí. Además, la producción circular
local puede ayudar a construir una mayor resiliencia a las escasez de
materiales críticos. En cuanto a los patrones temporales, los sistemas
de suministro circular pueden extender los horizontes de tiempo en los
que la información del producto y del material puede viajar. Las
tecnologías como blockchain pueden actuar como un puente hacia el
futuro, fomentando que los actores del presente piensen en cómo hablar
con los sistemas de suministro circular en ese tiempo lejano.\\

La ayuda de los socios que monitorean diversas escalas también puede
extender los horizontes espaciales y temporales considerados. Los
actores podrían ser más propensos a tener en cuenta los impactos
distantes en la biosfera y las escaseces críticas de materiales si esa
información es monitoreada y escalada hacia indicadores útiles para la
toma de decisiones a nivel de la empresa. Además, podrían tener la
información necesaria para sincronizar mejor sus flujos de materiales
con los ciclos regenerativos y/o absorbentes de la biosfera.

\hypertarget{la-informaciuxf3n-por-suxed-sola-no-es-suficiente}{%
\subsection{La información por sí sola no es
suficiente}\label{la-informaciuxf3n-por-suxed-sola-no-es-suficiente}}

En este apartado, en resumen, aunque los flujos de información son
necesarios para la aparición de redes y sistemas circulares de
suministro, no son suficientes por sí solos. Los actores necesitan tener
los recursos físicos y las capacidades para adaptarse a esa información,
lo que incluye máquinas de producción, sistemas de tecnología de la
información, vehículos de transporte y personal con las habilidades para
utilizar esos recursos. Además, se necesitan recursos financieros para
financiar todo ello. Es importante tener la voluntad de adaptarse
constructivamente a las señales de información recibidas, lo que puede
requerir nuevas lógicas e incentivos institucionales.

\hypertarget{conclusiuxf3n}{%
\section{Conclusión}\label{conclusiuxf3n}}

En este capítulo se buscó conceptualizar cómo la adaptación autónoma a
las señales de información puede llevar a la aparición de sistemas
circulares de suministro. La información sobre las redes de suministro
locales, sistemas de suministro extendidos e impactos en la biosfera es
particularmente importante para la producción circular. Se discutieron
las limitaciones de esta teoría y se sugirieron direcciones para futuras
investigaciones empíricas, como estudiar cómo las dinámicas sociales,
políticas, institucionales y financieras impactan la capacidad
adaptativa en los sistemas de suministro circular.

%
% ---- Bibliography ----



\end{document}
