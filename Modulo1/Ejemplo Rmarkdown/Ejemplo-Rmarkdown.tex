% This is samplepaper.tex, a sample chapter demonstrating the
% LLNCS macro package for Springer Computer Science proceedings;
% Version 2.21 of 2022/01/12
%
\documentclass[runningheads]{llncs}
%
\usepackage[T1]{fontenc}
% T1 fonts will be used to generate the final print and online PDFs,
% so please use T1 fonts in your manuscript whenever possible.
% Other font encondings may result in incorrect characters.


% tightlist command for lists without linebreak
\providecommand{\tightlist}{%
  \setlength{\itemsep}{0pt}\setlength{\parskip}{0pt}}




\usepackage{graphicx}
% Used for displaying a sample figure. If possible, figure files should
% be included in EPS format.
%
% If you use the hyperref package, please uncomment the following two lines
% to display URLs in blue roman font according to Springer's eBook style:
\usepackage{hyperref}
\usepackage{color}
\renewcommand\UrlFont{\color{blue}\rmfamily}


\begin{document}


\title{Ejemplo Simple Rmarkdown}
%
\titlerunning{Short title}
% If the paper title is too long for the running head, you can set
% an abbreviated paper title here
%
\author{Carolina
Fernández\inst{}\orcidID{0009-0004-4219-1041} \and Julián
Hidalgo\inst{}\orcidID{0009-0002-0617-6800} \and Ian
Johnson\inst{}\orcidID{0009-0001-2786-9385}}


\authorrunning{Microsoft}
% First names are abbreviated in the running head.
% If there are more than two authors, 'et al.' is used.
%

\institute{Universidad Nacional de Cuyo, Mendoza, Argentina\\
\email{\href{mailto:direccion.alumnos@ingenieria.uncuyo.edu.ar}{\nolinkurl{direccion.alumnos@ingenieria.uncuyo.edu.ar}}}\\}

\maketitle              % typeset the header of the contribution
%
\begin{abstract}
Sección del primer módulo de administración en la nube de Microsoft

\keywords{Microsoft \and Azure \and Nube}

\end{abstract}

\hypertarget{muxf3dulo-1}{%
\section{Módulo 1}\label{muxf3dulo-1}}

\hypertarget{introducciuxf3n-a-la-informuxe1tica-en-la-nube}{%
\subsection{\texorpdfstring{\emph{Introducción a la informática en la
nube}}{Introducción a la informática en la nube}}\label{introducciuxf3n-a-la-informuxe1tica-en-la-nube}}

La informática en la nube es la presentación de servicios informáticos a
través de internet. Estos incluyen infraestructura en TI común (máquinas
virtuales, almacenamiento, bases de datos y redes) y amplían las ofertas
de TI tradicionales sumando valores como Internet de las cosas (IoT),
aprendizaje automático (ML) e inteligencia artificial (IA). Si la
informática en la nube necesita aumentar rápidamente la infraestructura
de TI, no tiene que esperar a crear un centro de datos, puede usar la
nube para expandir rápidamente la superficie de TI. Básicamente, la
informática en la nube funciona como un medio para ejecutar un software
eligiendo potencia y características necesarias para el mismo. Este
equipo se encuentra en un centro de datos de un proveedor de nube (no
físicamente). Se pagan por los servicios únicamente sin necesidad de
realizar mantenimiento al equipo (lo realiza otra persona). Las ofertas
de los proveedores varían entre sí, pero los servicios básicos que todos
otorgan son la potencia de proceso y almacenamiento. Se definirán a
continuación para entender de qué se trata:

\begin{itemize}
\tightlist
\item
  Potencia de proceso: volumen de proceso que puede asumir el equipo,
  por ejemplo la cantidad de RAM y el procesador más reciente que
  tendría un equipo físico. Con la informática de nube, puede añadir o
  quitar potencia según los procesos que realice ahorrando costos (paga
  los recursos que usa).
\item
  Almacenamiento: volumen de datos que puede guardar el equipo, por
  ejemplo un equipo tradicional tiene una cantidad de espacio limitado
  que puede agotarse con el tiempo. Con la información en la nube, puede
  solicitar más almacenamiento según las necesidades y los proveedores
  se encargan del mantenimiento del equipo (copias de seguridad, sistema
  operativo actualizado, correcto funcionamiento, etc.).
\end{itemize}

\hypertarget{descripciuxf3n-del-modelo-de-responsabilidad-compartida}{%
\subsection{\texorpdfstring{\emph{Descripción del modelo de
responsabilidad
compartida}}{Descripción del modelo de responsabilidad compartida}}\label{descripciuxf3n-del-modelo-de-responsabilidad-compartida}}

Con el modelo de responsabilidad compartida, se separan las
responsabilidades entre el proveedor de servicios y el consumidor. La
seguridad física, alimentación, refrigeración y conectividad de red son
responsabilidad del proveedor (el consumidor no tiene acceso al centro
de datos). El consumidor es el responsable de los datos e información
almacenados en la nube y de la seguridad de acceso. Si usa una base de
datos SQL en la nube, el proveedor de servicios será el responsable de
mantener la base de datos real. Pero sigue siendo responsabilidad del
consumidor que los datos se ingieran en la base de datos. Si implementa
una máquina virtual con una base de datos SQL en ella, será el
responsable de las revisiones y actualizaciones de la base de datos,
mantenimiento de los datos e información almacenadas en ella. Con un
centro de datos local será el responsable de todo. El modelo de
responsabilidad compartida está vinculado a los tipos de servicio en la
nube: infraestructura como servicio (IaaS), plataforma como servicio
(PaaS) y software como servicio (SaaS). Se procede a desarrollar los
campos de estos servicios:

\begin{itemize}
\tightlist
\item
  Iaas: mayor responsabilidad en el consumidor mientras que el proveedor
  es responsable de los conceptos básicos de seguridad física, energía y
  conectividad.
\item
  Saas: mayor responsabilidad en el proveedor.
\item
  PaaS: se encuentra entre IaaS y SaaS, distribuyendo uniformemente la
  responsabilidad entre proveedor y consumidor.
\end{itemize}

En el siguiente diagrama se resaltan las responsabilidades de cada parte
en función del tipo de servicio en la nube:

\includegraphics[width=0.9\linewidth,]{Gráfico IaaS, SaaS, PaaS}

Responsabilidades del consumidor:

\begin{itemize}
\tightlist
\item
  Información y datos almacenados en la nube.
\item
  Dispositivos que se pueden conectar a la nube(teléfonos móviles,
  equipos, etc.).
\item
  Cuentas e identidades de las personas, servicios y dispositivos de la
  organización.
\end{itemize}

Responsabilidades del proveedor de nube:

\begin{itemize}
\tightlist
\item
  Centro de datos físicos.
\item
  Red física.
\item
  Hosts físicos.
\end{itemize}

El modelo de servicio definirá la responsabilidad de cosas como:

\begin{itemize}
\tightlist
\item
  Sistemas operativos.
\item
  Controles de red.
\item
  Aplicaciones.
\item
  Identidad e infraestructura.
\end{itemize}

%
% ---- Bibliography ----



\end{document}
